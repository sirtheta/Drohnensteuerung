Um mit dem Arduino die Game-Demo zu steuern, mussten wir ein entsprechendes Protokoll zum senden und empfangen der Daten implementieren. Mit dem Arduino werden über Serial.println die Daten als string versendet.\\

Das Protokoll haben wir wie folgt definiert:

\begin{itemize}
  \item Jeder Befehl, ob ausgehend oder eingehend startet mit vier chars, gefolgt von einer pipe ''|'' als Trennzeichen. Zum Abschluss des Befehl muss ein Semikolon '';' gesendet werden.
  \item Setzen der PID-Werte:
  \begin{itemize}
    \item Beispiel: \textbf{PIDS|X|P|0.3;}
    \item Der eingehende string muss mit PIDS starten. Das ''S'' steht für SET.
    \item Der zweite char spezifiziert für welche Achse der Wert gesetzt werden soll. X, Y oder Z.
    \item Der dritte char definiert welcher Wert, ob P, I oder D.
    \item Nach der letzten Pipe wird der zu setzende Wert übertragen.
  \end{itemize}
  \item  Anfordern von PID-Werten:
  \begin{itemize}
    \item Beispiel: \textbf{PIDR|X|I;}
    \item Der eingehende string muss mit PIDR starten. Das ''R'' steht für READ.
    \item Der zweite char spezifiziert für welche Achse der Wert gelesen werden soll. X, Y oder Z.
    \item Der dritte char definiert welcher Wert, ob P, I oder D.
  \end{itemize}
  \item  Zurücksenden der angeforderten PID-Daten:
  \begin{itemize}
    \item Beispiel: \textbf{PIDT|X|I|0.3;}
    \item Der ausgehende string muss mit PIDT starten. Das ''T'' steht für Transfer.
    \item Der zweite char spezifiziert für welche Achse der Wert gesendet wird. X, Y oder Z.
    \item Der dritte char definiert welcher Wert, ob P, I oder D.
    \item Nach der letzten Pipe wird der gelesene Wert übertragen. Dieser Wert muss beim Empfänger durch 10'000 gerechnet werden, da ein float über Serial.println auf zwei Kommastellen gekürzt wird und deshalb für den Transfer mit 10'000 multipliziert als int übergeben wird.
  \end{itemize}
  \vspace{3mm}
  \begin{lstlisting}
  // Send the requested PID Value via Serial
  void ProtocolHandler::sendPIDToSerial(float _fVal, 
                                        String _strAxis, 
                                        String _strPidParam)
  {
    int iVal = _fVal * 10000; // transfer float to int because serial (string) 
                              // will only print two digits after comma
    Serial.println(String(chrTransferPid) + 
                   String(paramSeparator) + 
                   _strAxis + 
                   String(paramSeparator) + 
                   _strPidParam + 
                   String(paramSeparator) + 
                   iVal + 
                   String(cmdTerminator));
  }
  \end{lstlisting}
\newpage
  \item Senden des Data-Frames zum Steuern der Game-Demo
  \begin{itemize}
    \item Beispiel: \textbf{DATT|90.05|90.2|3.5|0.1|0.01|1.05}
    \item Der ausgehende string muss mit DATT starten.
    \item Die drei ersten Werte liefern die Gyro-Achswerte für X, Y und Z.
    \item Die drei letzten Werte liefern die Accelerometer-Werte für X, Y und Z
    \item DATT|X|Y|Z|X|Y|Z
  \end{itemize}
  \vspace{3mm}
  \begin{lstlisting}
void ProtocolHandler::sendDataFrame(float _iData[], int _iDataLength)
{
  String frame = chrDataTransfer;

  for(int i =0; i < _iDataLength ;i++)
  {
    frame = frame + paramSeparator + String(_iData[i]);
  }

  Serial.println(frame);
}
  \end{lstlisting}
\end{itemize} 

\subsubsection{Datenempfang}
Um die Daten über die Serialschnittstelle empfangen zu können, wird in der loop()-Funktion geprüft ob eine Nachricht eingeht. Ist dies der Fall, wird mit Serial.read() jeder char der eingehenden Nachricht in einen Buffer abgefüllt und ausgewertet. Durch die genaue Definition des Protokolls kann dann der Befehl ausgewertet und anschliessend ausgeführt werden.

\vspace{3mm}
\begin{lstlisting}
// Check to see if anything is available in the serial receive buffer
while (Serial.available() > 0)
{
  // Create a place to hold the incoming message
  static char message[MAX_MESSAGE_LENGTH];
  static unsigned int message_pos = 0;

  // Read the next available byte in the serial receive buffer
  char inByte = Serial.read();

  // Message coming check for pipe
  if (inByte != paramSeparator && inByte != cmdTerminator &&
      message_pos < MAX_MESSAGE_LENGTH - 1)
  {
    // Add the incoming byte to message
    message[message_pos] = inByte;
    message_pos++;
  }
  else
  {
    // Adds terminator to message 
    message[message_pos] = '\0';
    // Ready for next parameter
    paramPosition++;

    // Reset for the next message
    message_pos = 0;

    // Fill the incoming message into the correct string for later execution
    switch (paramPosition)
    {
    case command:
      strCommand = String(message);
      break;
    case axis:
      strAxis = String(message);
      break;
    case pidParam:
      strPidParam = String(message);
      break;
    case value:
      fValue = String(message).toFloat();
      break;
    default:
      break;
    }

    if (inByte == cmdTerminator)
    {
      paramPosition = 0;        // Reset parameter position
      message_pos = 0;          // Reset message position, read for new message
      executeIncomingCommand(); // Execute the incoming command
    }
  }
}
\end{lstlisting}

\subsubsection{Senden der Daten}
Die PID-Daten werden je Achse und je PID-Term separat versendet. Um die Drohne im Game zu steuern, wollten wir dies ebenfalls so implementieren, dies führte jedoch zu nicht besonders flüssigem Spielverhalten. Aus diesem Grund haben wir uns entschieden, immer alle X,Y und Z- Werte auf einmal zu versenden.